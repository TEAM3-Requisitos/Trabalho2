\section{Planejamento}

A Sprint 1 teve por principal objetivo desenvolver as histórias de usuários classificadas como prioritárias pela cliente, sendo essas de maior valor para o contexto geral da aplicação.

Tendo tais histórias priorizadas, foi levantado junto do cliente os critérios que deveriam ser compreendidos para o pleno funcionamento e a aceitação e completude da história implementada, além da validação da pontuação pré-definida pela equipe.

Como resultado desta atividade de priorização, seguem as \textit{User Stories} a serem implementadas.

\FloatBarrier
\begin{table}[!htpd]
\centering
\caption{US1 - Priorizada}
\label{my-label}
\begin{tabular}{|l|}
\hline
Eu como usuário, quero cadastrar-me no sistema de compartilhamento de projetos.                                                                                                                                   \\ \hline
Épico 1 - Feature 1 - US1 - Sprint 1                                                                                                                                                                             \\ \hline
Pontos: 3                                                                                                                                                                                                         \\ \hline
\begin{tabular}[c]{@{}l@{}}Critérios de aceitação: \\        Usuário precisa preencher: \\             Nome de usuário; \\             Nome completo; \\             Senha de no mínimo 8 caracteres\end{tabular} \\ \hline
\end{tabular}
\end{table}
\FloatBarrier

\FloatBarrier
\begin{table}[!htpd]
\centering
\caption{US2 - Priorizada}
\label{my-label}
\begin{tabular}{|l|}
\hline
Eu como usuário, quero alterar meus dados cadastrais para manter meu cadastro atualizado.                                                                                                                                              \\ \hline
Épico 1 - Feature 1 - US2 - Sprint 1                                                                                                                                                                                                  \\ \hline
Pontos: 3                                                                                                                                                                                                                              \\ \hline
\begin{tabular}[c]{@{}l@{}}Critérios de Aceitação:\\        O usuário precisa ter acesso ao seus dados cadastrados, de maneira que ele,possa editá-los;\\        Usuário precisa salvar os dados alterados após a edição.\end{tabular} \\ \hline
\end{tabular}
\end{table}
\FloatBarrier

\FloatBarrier
\begin{table}[!htpd]
\centering
\caption{US3 - Priorizada}
\label{my-label}
\begin{tabular}{|l|}
\hline
Eu como usuário, quero excluir minha conta para não ser mais usuário do sistema.                                                                                                        \\ \hline
Épico 1 - Feature 1 - US3 - Sprint 1                                                                                                                                                    \\ \hline
Pontos: 3                                                                                                                                                                               \\ \hline
\begin{tabular}[c]{@{}l@{}}Critérios de Aceitação:\\        Deve haver uma confirmação de exclusão.\\        Usuário precisa ser redirecionado a tela de login do sistema.\end{tabular} \\ \hline
\end{tabular}
\end{table}
\FloatBarrier

\FloatBarrier
\begin{table}[!htpd]
\centering
\caption{US4 - Priorizada}
\label{my-label}
\begin{tabular}{|l|}
\hline
\begin{tabular}[c]{@{}l@{}}Eu como usuário, quero fazer login no sistema de compartilhamento de projetos \\ para acessar o sistema.\end{tabular}                                              \\ \hline
Épico 1 - Feature 2 - US4 - Sprint 1                                                                                                                                                         \\ \hline
Pontos: 4                                                                                                                                                                                     \\ \hline
\begin{tabular}[c]{@{}l@{}}Critérios de Aceitação:\\        Usuário precisa preencher:\\            Nome de usuário;\\            Senha\\        Em seguida deve efetuar o login\end{tabular} \\ \hline
\end{tabular}
\end{table}
\FloatBarrier

\FloatBarrier
\begin{table}[!htpd]
\centering
\caption{US5 - Priorizada}
\label{my-label}
\begin{tabular}{|l|}
\hline
\begin{tabular}[c]{@{}l@{}}Eu como usuário, quero fazer logout no sistema para sair do sistema e encerrar \\ a sessão.\end{tabular}                                                                     \\ \hline
Épico 1 - Feature 2 - US5 - Sprint 1                                                                                                                                                                    \\ \hline
Pontos: 3                                                                                                                                                                                               \\ \hline
\begin{tabular}[c]{@{}l@{}}Critérios de Aceitação:\\        É preciso ser sempre de fácil acesso a opção de logout;\\        Usuário precisa ser redirecionado a tela de login do sistema.\end{tabular} \\ \hline
\end{tabular}
\end{table}
\FloatBarrier

\FloatBarrier
\begin{table}[!htpd]
\centering
\caption{US8 - Priorizada}
\label{my-label}
\begin{tabular}{|l|}
\hline
\begin{tabular}[c]{@{}l@{}}Eu como criador de projeto, quero criar um projeto para para que ele possa ser \\ desenvolvido no sistema.\end{tabular}                                                                                                                                                                                                                                                                                                                                                                                                                                                                                    \\ \hline
Épico 2 - Feature 3 - US8 - Sprint 1                                                                                                                                                                                                                                                                                                                                                                                                                                                                                                                                                                                                  \\ \hline
Pontos: 4                                                                                                                                                                                                                                                                                                                                                                                                                                                                                                                                                                                                                             \\ \hline
\begin{tabular}[c]{@{}l@{}}Crtiérios de Aceitação:\\      O usuário precisa informar:\\            Título do projeto;\\            Categoria;\\            Dificuldade;\\            O status que o projeto se encontra.\\        É necessário:\\            Adicionar uma imagem ao projeto;\\            Adicionar uma breve descrição.\\        Em seguida deve-se confirmar a criação do projeto\\        Logo após criar o projeto, o usuário deve ser redirecionado a tela de \\        visualização de projetos.\end{tabular} \\ \hline
\end{tabular}
\end{table}
\FloatBarrier

\FloatBarrier
\begin{table}[!htpd]
\centering
\caption{US9 - Priorizada}
\label{my-label}
\begin{tabular}{|l|}
\hline
\begin{tabular}[c]{@{}l@{}}Eu como criador do projeto, quero remover um projeto para interromper/impedir \\ o seu desenvolvimento no sistema.\end{tabular}                                                                                                                                                                       \\ \hline
Épico 2 - Feature 3 - US9 - Sprint 1                                                                                                                                                                                                                                                                                             \\ \hline
Pontos: 4                                                                                                                                                                                                                                                                                                                        \\ \hline
\begin{tabular}[c]{@{}l@{}}Critérios de Aceitação: \\        É necessário que o usuário esteja na tela de visualização de projetos;\\        O usuário precisa selecionar o projeto desejado;\\        Deve haver uma confirmação de exclusão;\\        O usuário deve retornar a tela de visualização de projetos.\end{tabular} \\ \hline
\end{tabular}
\end{table}
\FloatBarrier

\FloatBarrier
\begin{table}[!htpd]
\centering
\caption{US10 - Priorizada}
\label{my-label}
\begin{tabular}{|l|}
\hline
\begin{tabular}[c]{@{}l@{}}Eu como criador de projeto, quero editar os dados do projeto para corrigir algum \\ erro de digitação, ou refinar os dados cadastrais do projeto.\end{tabular} \\ \hline
Épico 2 - Feature 3 - US10 - Sprint 1                                                                                                                                                     \\ \hline
Pontos: 4                                                                                                                                                                                 \\ \hline
\begin{tabular}[c]{@{}l@{}}Critérios de Aceitação:\\        O usuário precisa ter acesso ao seus dados cadastrados no projeto, \\        de maneira que possa editá-los.\end{tabular}     \\ \hline
\end{tabular}
\end{table}
\FloatBarrier

\FloatBarrier
\begin{table}[!htpd]
\centering
\caption{US33 - Priorizada}
\label{my-label}
\begin{tabular}{|l|}
\hline
\begin{tabular}[c]{@{}l@{}}Eu como criador do projeto, quero manter tarefas para que os contribuidores \\ possam contribuir com elas.\end{tabular}                                                                                                                                                                                                                                                                                                                                                                                            \\ \hline
Épico 2 - Feature 11 - US33 - Sprint 1                                                                                                                                                                                                                                                                                                                                                                                                                                                                                                                              \\ \hline
Pontos: 4                                                                                                                                                                                                                                                                                                                                                                                                                                                                                                                                     \\ \hline
\begin{tabular}[c]{@{}l@{}}Critérios de Aceitação: \\        A tarefa deve ser adicionada no momento de criação do projeto ou no \\        momento de edição do projeto;\\        No formulário de cadastro de projeto o usuário deverá selecionar a opção \\        “Adicionar Tarefa”\\        Em seguida deve preencher os campos do formulário de tarefa:\\            Título;\\            Status;\\            Dificuldade;\\            Descrição.\\         A tarefa deve ficar visível no projeto e no menu de tarefas.\end{tabular} \\ \hline
\end{tabular}
\end{table}
\FloatBarrier

\section{Desenvolvimento}

A fase de Desenvolvimento iniciou-se com a reunião de planejamento da sprint, nesta reunião efetuou-se a priorização e pontuação das histórias escolhidas pelo product owner juntamente com o time de desenvolvimento. Com isso os itens do product backlog passaram para o Sprint backlog e iniciou-se o processo de implementação das US presentes no Sprint backlog.

O time teve dificuldades em entender o protótipo anteriormente criado pelos integrantes d Eletrojun, o time então decidiu buscar informações com os integrantes da empresa que haviam iniciado o projeto de desenvolvimento. Os integrantes da empresa sanaram as dúvidas que persistiam e então foi possível realizar a implementação/correção das histórias US01, US02, US03, US04, US05, US08, US09, US10. A US33 foi parcialmente concluída entretanto os critérios de aceitação 8 e 9 não foram satisfeitos.

Na reunião de retrospectiva da Sprint definiu-se que a US33, que não foi devidamente concluída, for remanejada para a próxima Sprint, entrando como item de dívida técnica.
