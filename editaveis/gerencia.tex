\section {Nível de Portfólio}

A divisão em camadas do SAFe proporciona uma visão ampla de todos os níveis de requisitos que devem ser tratados. A camada Portifólio abrange os requisitos a nível de negócio, ou seja, uma visão com alto nível de abstração. De acordo com o nosso processo, o nível de portifólio visto na (Figura \ref{img:portfolio}) possui as seguintes tarefas:

\begin{itemize}
\item Analisar a empresa Eletrojun.
\item Compreender as necessidades da empresa.
\item Identificar conjunto de épicos.
\item Priorizar épico.
\item Gerenciar épicos.
\end{itemize}

\FloatBarrier
\begin{figure}[!htpd]
		\centering
		\includegraphics[scale=0.4]{figuras/portfolio}
		\label{img:portfolio}
		\caption{Processo à nível de Portfólio}
\end{figure}

As tarefas listadas foram cumpridas com o auxílio das técnicas de elicitação, Brainstorm e Entrevista. As atas de reunião, juntamente com as entrevistas, podem ser encontradas no apêndice XX.

\subsection {Épicos Identificados}

\textbf{Tema de Investimento:} Gestão e desenvolvimento de projetos.

 A partir das prioridades da organização, identificou-se que a empresa desejava investir na Gerência e Desenvolvimento de novos projetos. Este tema de investimento tem a função de fomentar o desenvolvimento de novos projetos por parte dos alunos da FGA, tendo como principal característica a necessidade de um meio que una os estudantes e crie um ambiente favorável para o desenvolvimento de ideias e projetos.

Para a especificação dos épicos utilizou-se o padrão recomendado pelo SAFe [SAFe 2015], que é o template  lightweight business case.

\textbf{Épico 01 - EP-01:} Gerenciamento de Usuários: Abrange todas as ações de administração dos usuários do sistema, desde o cadastro de um usuário até a sua exclusão do sistema.


\textbf{Épico 02 - EP-02:} Gerenciamento de Projetos: Abrange todas as ações de administração de projetos no sistema, desde a sua publicação até o seu cumprimento total;
\textbf{Épico 03 - EP-03:} Gerenciamento de Atividades: Abrange todas as ações de administração e gerenciamento de atividades pontuais, dentro e fora dos projetos.

\textbf{Épico 04 - EP-04:} Gerenciamento de Premiações: Abrange todas as ações manutenção de premiações aos usuários do sistema.

\section {Nível de Programa}
A camada de programa é a camada intermediária do processo. Este nível é responsável por identificar requisitos concretos e estabelecer estratégias para a implementação da solução. O nível de programa (Figura \ref{img:programa}) possui as seguintes atividades:

\begin{itemize}
\item Levantar Features
\item Identificar requisitos não-funcionais
\item Definir Roadmap
\item Priorizar Features
\item Planejamento da Release
\item Gerenciar Features
\item Planejamento da Release
\end{itemize}

\FloatBarrier
\begin{figure}[!htpd]
		\centering
		\includegraphics[scale=0.4]{figuras/programa}
		\label{img:programa}
		\caption{Processo à nível de Programa}
\end{figure}

\subsection{Requisitos não-funcionais}

\textbf{Requisitos de portabilidade:} O sistema deverá rodar em dispositivos móveis e também em Computadores. Os navegadores suportados devem incluir Google Chrome, Internet Explorer 8 ao 11, Mozilla Firefox, Ópera e Safari.

\textbf{Requisitos de implementação:} O sistema deverá ser desenvolvido na linguagem Ruby com Framework Rails.

\textbf{Requisitos de eficiência:} O sistema deverá processar todas as requisições dos usuários ao mesmo tempo e sem demora .

\textbf{Requisitos de confiabilidade:} O sistema deverá ter alta disponibilidade, ficando disponível 24 horas por dia e todos os dias da semana.

\subsection{Features Identificadas}
Com base nas atividades descritas no processo, a equipe de Engenharia de Requisitos levantou as Features junto a cliente, para que a partir destas Features as Histórias de Usuário possam ser levantadas.

\textbf{Feature 1 (EP-1 FT-1):} Manutenção de Usuários
Esta feature tem como objetivo manter os usuários do sistema, permitindo o cadastro e edição de perfil.

\textbf{Feature 2 (EP-1 FT-2):} Acesso dos Usuários
Esta feature é responsável pelo controle de acesso dos usuários, que podem realizar o login e o logout.

\textbf{Feature 3 (EP-2 FT-3):} Manutenção de projetos
Feature responsável pelo cadastro, edição, e exclusão de projetos.

\textbf{Feature 4 (EP-2 FT-4):} Manutenção de usuários em projetos
Feature responsável por incluir e administrar usuários em um projeto.

\textbf{Feature 5 (EP-4 FT-5):} Sistema de pontuações e níveis para usuário.
Feature responsável por permitir que o usuário possa obter pontuação nos projetos e desta forma avançar de nível na medida em que se contribui.

\textbf{Feature 6 (EP-2 FT-6):} Sistema de avaliação e ranking de projeto
Feature responsável por permitir que o usuário possa avaliar projetos, bem como atualização do ranking de projetos.

\textbf{Feature 7 (EP-3 FT-7):} Sistema de ajuda
Feature responsável por permitir que o usuário peça ajuda aos outros usuários.

\textbf{Feature 8 (EP-3 FT-8):} Sistema de registro de vendas
Feature responsável por permitir que o usuário registre a venda de um produto criado na plataforma.

\textbf{Feature 9 (EP-1 FT-9):} Sistema de adicionar e seguir usuários
Feature responsável por permitir que o usuário adicione e siga amigos e projetos.

\textbf{Feature 10 (EP-3 FT-10):} Sistema de pesquisa
Feature responsável por permitir que o usuário pesquise por outros usuários e projetos.

\textbf{Feature 11 (EP-2 FT-11):} Sistema de tarefas
Feature responsável por especificar e controlar atividades designadas aos usuários do projeto.

\textbf{Feature 12 (EP-3 FT-12):} Sistema de notificações
Feature responsável por manter sistema de recebimento e envio de notificações por parte de usuários e por parte do sistema.

\textbf{Feature 13 (EP-3 FT-13):} Sistema de comunicações
Feature responsável por manter sistema de comunicações entre usuários e entre sistema e usuários.

\textbf{Feature 14 (EP-1 FT-14):} Opções de configurações e preferências
Feature responsável por manter as opções de configurações da conta e preferências do usuário.

\textbf{Feature 15 (EP-4 FT-15):} Premiações em moedas de acordo com contribuições
Feature responsável por manter o sistema de premiações e bonificações de usuários conforme seu merecimento.

\textbf{Feature 16 (EP-1 FT-16):} Administração do Sistema
Feature responsável pela parte administrativa do sistema, onde o administrador pode monitorar e cancelar contas de usuários.

\subsection{Roadmap}

Roadmap constitui-se em um mapa baseado em tempo, composto por camadas. O método, por flexível, apresenta múltiplos escopos, tendo, por consequência, distintas formas de representação [1].

Em geral, Roadmaps são utilizados para estabelecer um plano ou estratégia para atingir metas. Na arquitetura de software, esse tipo de plano ou estratégia detalha o conjunto de atividades de trabalho relacionados a arquitetura e estabelece prazos de entrega na linha do tempo de sua produção, com objetivo de evidenciar como se dará a evolução do trabalho.

Leffingwell é ainda mais pontual, descrevendo o Roadmap como uma série de releases planejadas em datas, onde cada uma delas possui uma lista de features priorizadas.[2]

No presente projeto, foi levado em conta a assincronicidade das features, por terem seu desenvolvimento distribuído entre 2 sprints, de forma a permitir a possibilidade de sua entrega em duas parcelas. Entretanto, tal metodologia visa a completude de cada entrega, sendo assim a primeira parcela completamente independente da segunda, no que diz respeito a sua plena funcionalidade.

Desta forma, o Roadmap proposto apresenta-se da seguinte forma:

IMAGEMMMMMMMMMM

Como descrito acima, a Feature 2 terá uma parcela entregue na Sprint 1 e uma outra parcela entregue na Sprint 2, garantindo a plena usabilidade das duas parcelas.
O Roadmap completo se encontra no apêndice XX.

\section{Nível de Time}

O nível de time compreende a camada mais baixa de todo o processo ágil, esta camada é responsável pela implementação da solução técnica, e também pelo detalhamento mais estrito dos requisitos levantados nas camadas superiores, gerando assim as histórias de usuário. O nível de time (Figura \ref{img:programa}) são desempenhadas as seguintes atividades:

\begin{itemize}
\item Levantar User Stories
\item Planejar Sprint
\item Priorizar e detalhar User Stories
\item Desenvolver Sprint
\item Retrospectiva da Sprint
\item Gerenciar User Stories
\end{itemize}

\FloatBarrier
\begin{figure}[!htpd]
		\centering
		\includegraphics[scale=0.4]{figuras/time}
		\label{img:time}
		\caption{Processo à nível de Time}
\end{figure}

\subsection{User Stories Identificadas}

Logo abaixo é possível verificar as histórias dos usuários, que estão organizadas nos respectivos épicos e features. As histórias seguem o padrão do cartão abaixo.

TABELAAA

Atores identificados:

\begin{itemize}
\item Usuário do sistema
\item Usuário colaborador
\item Gerente de projeto
\item Administrador do Sistema
\end{itemize}

\section{Gerência de Mudança}

\subsection{Atributos de Requisitos}

De forma a contribuir na identificação e na obtenção de informações mais detalhadas dos requisitos dentro do projeto, foi realizada uma identificação por atributos nestes, dentro da plataforma, de forma a identificar rastreabilidade, progresso, prioridade e risco dentro do projeto.

\subsubsection{Origem}

De forma a garantir a rastreabilidade e origem dos requisitos, os atributos foram identificados de acordo com a seguinte tabela:

\FloatBarrier
\begin{table}[\htp]
\centering
\caption{Atributos do requisito}
\label{my-label}
\begin{tabular}{|l|l|l}
\cline{1-2}
EP & Épico      &  \\ \cline{1-2}
FT & Feature    &  \\ \cline{1-2}
US & User Story &  \\ \cline{1-2}
\end{tabular}
\end{table}

\FloatBarrier
\begin{figure}[!htpd]
		\centering
		\includegraphics[scale=0.4]{figuras/origem}
		\label{img:origem}
		\caption{Atributo do requisito - Origem}
\end{figure}

\subsubsection{Status}

Visando monitorar o grau de completude do requisito, foi utilizado um atributo de Status do requisito, em forma de porcentagem.

\FloatBarrier
\begin{figure}[!htpd]
		\centering
		\includegraphics[scale=0.4]{figuras/status}
		\label{img:status}
		\caption{Atributo do requisito - Status}
\end{figure}

\subsubsection{Prioridade}

Para caracterizar a prioridade dos requisitos, eles foram classificado entre prioridade média, alta ou baixa, de forma a evidenciar sua importância no contexto do projeto.

\FloatBarrier
\begin{figure}[!htpd]
		\centering
		\includegraphics[scale=0.4]{figuras/prioridade}
		\label{img:prioridade}
		\caption{Atributo do requisito - Prioridade}
\end{figure}

\subsubsection{Complexidade}

De forma a ter um controle de tempo de entrega dos requisitos, foi atribuído ainda o nível de complexidade deste, de forma a permitir um maior controle de data de entrega. Tal classificação tem 3 níveis: Baixa, Média e Alta. É feita a atribuição na descrição do requisito.

\subsubsection{Risco}

Caracteriza o risco que a implementação deste requisito traz para a integridade do projeto. Também classificado em 3 níveis: Baixo, Médio e Alto e também atribuído na descrição do requisito.

\FloatBarrier
\begin{figure}[!htpd]
		\centering
		\includegraphics[scale=0.4]{figuras/risco}
		\label{img:risco}
		\caption{Atributo do requisito - Risco}
\end{figure}

\section{Rastreabilidade}

Rastreabilidade define-se, segundo Edwards, como sendo a técnica usada para prover relacionamento entre requisitos, arquitetura e implementação final do sistema [3]. Ela auxilia ainda na compreensão dos relacionamentos existentes entre requisitos do software ou entre artefatos de requisitos, arquitetura e implementação. Esses relacionamentos permitem aos projetistas mostrar que o projeto atende aos requisitos. A rastreabilidade também apóia a detecção precoce daqueles requisitos não atendidos pelo software [4].

IMAGEMMM

A rastreabilidade foi foi documentada na ferramenta Tracecloud como se segue:

\FloatBarrier
\begin{figure}[!htpd]
		\centering
		\includegraphics[scale=0.4]{figuras/rast1}
		\label{img:rast1}
		\caption{Rastreabilidade dos Requisitos (Épicos - Features)}
\end{figure}

\FloatBarrier
\begin{figure}[!htpd]
		\centering
		\includegraphics[scale=0.25]{figuras/rast2}
		\label{img:rast2}
		\caption{Rastreabilidade dos Requisitos (Features - User Stories)}
\end{figure}

\FloatBarrier
\begin{figure}[!htpd]
		\centering
		\includegraphics[scale=0.25]{figuras/rast3}
		\label{img:rast3}
		\caption{Rastreabilidade dos Requisitos completa (Features - User Stories)}
\end{figure}
