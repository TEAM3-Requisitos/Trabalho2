“Os problemas que os engenheiros de software têm para solucionar são, muitas vezes, imensamente complexos. Compreender a natureza dos problemas pode ser muito difícil, especialmente se o sistema for novo. Consequentemente, é difícil estabelecer com exatidão o que o sistema deve fazer. As descrições das funções e das restrições são os requisitos para o sistema; e o processo de descobrir, analisar, documentar e verificar essa funções e restrições é chamado de engenharia de requisitos.” \cite{ian}.

Este documento irá apresentar a execução do processo de Engenharia de Requisitos conforme planejado no Trabalho 1. Para esta segunda fase  do trabalho foi elaborado um novo cronograma que pode ser encontrado no apêndice \ref{sec:cronograma}.

A abordagem utilizada neste processo foi a ágil,  seguindo o modelo do \textit{Scaled Agile Framework} (SAFe) e a modelagem deste processo encontra-se no apêndice \ref{sec:processo}. A ferramenta para gerência dos requisitos foi o Tracecloud e as técnicas de elicitação utilizadas foram a entrevista e  o brainstorming.
