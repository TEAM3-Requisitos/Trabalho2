Esta seção traz informações a respeito da experiência da equipe na execução do trabalho, bem como as obtidas com as Técnicas de Elicitação de Requisitos utilizadas e as considerações finais a respeito do processo de aprendizagem da disciplina.

\section{Experiência com as Técnicas de Elicitação:}

Para o presente trabalho, as técnicas de elicitação de requisitos adotadas foram de suma importância para o pleno entendimento do problema a ser solucionado e o levantamento dos requisitos em sua completude, de forma que fossem satisfeitas as necessidades, bem como os problemas que o mesmo buscava sanar com a solução proposta.

Inicialmente foi utilizada, assim como planejada anteriormente, a técnica de entrevista. Foram definidas agendas, de forma a se seguir um roteiro previamente combinado com o cliente, para que fosse obtido os insumos e produtos necessários para o levantamento adequado dos requisitos da solução.

A entrevista na fase inicial do projeto foi de extrema importância para a aproximação entre a equipe de Engenharia de Requisitos e o cliente, garantindo uma boa compreensão do contexto do problema bem como a proposta de solução idealizada. Nesta técnica foi compreendido ainda a solução inicializada previamente e os desacordos e acertos desta para com as reais necessidades do cliente, demonstrando aí a aptidão da cliente para um nível mais técnico e profundo de diálogo no que diz respeito a implementação da solução.

Porém, fugindo do planejamento inicial, foi constatado que os diálogos mantidos nas entrevistas ainda não estavam sendo suficientes para que fossem elicitados e compreendidos todos os requisitos da aplicação, de forma que se fez necessária a utilização da técnica \textit{Brainstorming}, onde o cliente era incentivado a manifestar soluções imaginadas para a aplicação, passando pela fase da geração de ideias, em seguida o esclarecimento do processo proposto e finalmente a avaliação de tal proposta, podendo assim ser informalmente documentada para a geração de requisitos consistentes para a plataforma solução.

Foi constatado desta maneira que a utilização das duas técnicas proporcionaram uma maior dinamicidade e assim uma forma mais contundente de se levantar as reais necessidades e desejos do cliente para o contexto da aplicação, de forma ainda mais rápida e precisa.
