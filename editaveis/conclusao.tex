A partir da definição da abordagem e consequentemente elaboração do processo de engenharia de requisitos a equipe de engenharia de requisitos passou a estar focada nas atividades de elicitação de requisitos. Para a extração dos requisitos a equipe ER desempenhou uma série da tarefas juntamente ao Product Owner, ao realizar estas tarefas a equipe de ER obteve experiencia na extração dos requisitos, descobriu e experimentou técnicas desconhecidas por parte dos integrantes da equipe.  É necessário salientar que o product owner do projeto, a diretora da eletrojun Mônica Damasceno também pode adquirir conhecimento visto que como PO, desempenhou um papel de extrema importância no projeto.

Durante o desenvolvimento do projeto, a equipe ER seguiu a metodologia ágil, o uso deste metodologia facilitou a elicitação e o desenvolvimento/implementação da solução proposta. A diretora tinha um bom relacionamento com os integrantes da equipe, isto possibilitou que o trabalho seguisse um fluxo contínuo, não houveram problemas sociais e de relacionamento em grupo.

Os principais obstáculos e dificuldades foram na etapa de definição do escopo do projeto. Com o andamento das reuniões e entrevistas, o número de Features e Histórias de usuário cresciam de forma descontrolada. O projeto inicial obteve um aumento significativo, essas mudanças exigiram um aumento de trabalho significativo na gerencia de mudanças.

\section{Experiência de Execução do Trabalho}
	O trabalho de Engenharia de Requisitos é fundamental o desenvolvimento de várias habilidades no que diz respeito a elicitação de requisitos e reuniões com clientes. As reuniões com o cliente foram significativas no aprendizado dos alunos. A necessidade da correta extração dos requisitos é latente, muitos projetos de software falham por estarem com os requisitos incompletos, inexistentes ou errados. É por este motivo que se deve ter uma preocupação maior quanto esta atividade. Os alunos observaram e descobriram os desafios que existem nesta atividade, e buscaram soluções para vencer os obstáculos que apareceram.

\section{Experiência com as Técnicas de Elicitação}

Para o presente trabalho, as técnicas de elicitação de requisitos adotadas foram de suma importância para o pleno entendimento do problema a ser solucionado e o levantamento dos requisitos em sua completude, de forma que fossem satisfeitas as necessidades, bem como os problemas que o mesmo buscava sanar com a solução proposta.

Inicialmente foi utilizada, assim como planejada anteriormente, a técnica de entrevista. Foram definidas agendas, de forma a se seguir um roteiro previamente combinado com o cliente, para que fosse obtido os insumos e produtos necessários para o levantamento adequado dos requisitos da solução.

A entrevista na fase inicial do projeto foi de extrema importância para a aproximação entre a equipe de Engenharia de Requisitos e o cliente, garantindo uma boa compreensão do contexto do problema bem como a proposta de solução idealizada. Nesta técnica foi compreendido ainda a solução inicializada previamente e os desacordos e acertos desta para com as reais necessidades do cliente, demonstrando aí a aptidão da cliente para um nível mais técnico e profundo de diálogo no que diz respeito a implementação da solução.

Porém, fugindo do planejamento inicial, foi constatado que os diálogos mantidos nas entrevistas ainda não estavam sendo suficientes para que fossem elicitados e compreendidos todos os requisitos da aplicação, de forma que se fez necessária a utilização da técnica \textit{Brainstorming}, onde o cliente era incentivado a manifestar soluções imaginadas para a aplicação, passando pela fase da geração de ideias, em seguida o esclarecimento do processo proposto e finalmente a avaliação de tal proposta, podendo assim ser informalmente documentada para a geração de requisitos consistentes para a plataforma solução.

Foi constatado desta maneira que a utilização das duas técnicas proporcionaram uma maior dinamicidade e assim uma forma mais contundente de se levantar as reais necessidades e desejos do cliente para o contexto da aplicação, de forma ainda mais rápida e precisa.

\section{Execução da Disciplina}

A disciplina proporcionou aos alunos o descobrimento do universo que existe na definição dos requisitos. A disciplina  engenharia de requisitos é um curso completo, ele abrange todas as etapas deste processo. A equipe de ER percorreu um longo caminho até chegar ao final desse processo, e entre as etapas do processo pode-se perceber o objetivo de cada etapa, vivenciando na teoria e prática o desenvolvimento do projeto.

As mudanças adotadas na disciplina como a mudança na escolha dos clientes foi positiva. Os grupos puderam trabalhar com as empresas júnior do próprio campus, isso fortaleceu as relações entre os alunos e as empresas, além de agregar valor aos projetos desenvolvidos no próprio campus.
